\documentclass[brudnopis]{xmgr}

\usepackage{listings}
% "define" Scala
\lstdefinelanguage{scala}{morekeywords={class,object,trait,extends,with,new,if,while,for,def,val,var,this},
otherkeywords={->,=>},
sensitive=true,
morecomment=[l]{//},
morecomment=[s]{/*}{*/},
morestring=[b]"}
% Default settings for code listings
\lstset{frame=tb,language=scala,aboveskip=3mm,belowskip=3mm,showstringspaces=false,columns=flexible,basicstyle={\small\ttfamily}}


%\defaultfontfeatures{Scale=MatchLowercase}
%\setmainfont[Numbers=OldStyle,Ligatures=TeX]{Minion Pro}
%\setsansfont[Numbers=OldStyle,Ligatures=TeX]{Myriad Pro}
% for fontspec version < 2.0
\setmainfont[Numbers=OldStyle,Mapping=tex-text]{Minion Pro}
\setsansfont[Numbers=OldStyle,Mapping=tex-text]{Myriad Pro}
%\setmonofont[Scale=0.75]{Monaco}

% Opcjonalnie identyfikator dokumentu 
% drukowany tylko z włączoną opcją 'brudnopis':
\wersja   {wersja wstępna [\ymdtoday]}

\author   {Mateusz Szygenda}
\nralbumu {186\,436}
\email    {mateusz.szygenda@gmail.com}

\title    {Wykorzystanie baz grafowych w języku Scala}
\date     {2014}
\miejsce  {Gdańsk}

\opiekun  {dr Wiesław Pawłowski}

% dodatkowe polecenia
%\renewcommand{\filename}[1]{\texttt{#1}}

\begin{document}

\begin{abstract}
Wstęp
\end{abstract}
\keywords{Scala, neo4j, DSL, języki domenowe, bazy grafowe}

% tytuł i spis treści
\maketitle

%
% wstęp
\introduction

Celem pracy było stworzenie narzędzia dla języka Scala pozwalającego na efektywne wykorzystanie jednej z opisywanych baz grafowych. Język ten został wybrany ze względu na dostępność licznych mechanizmów umożliwiających definiowanie tzw. języków domenowych (DSL).

\chapter{Bazy grafowe}

\section{Wstęp}

\section{Rodzaje baz grafowych}

\section{Wiodące bazy}

\section{Neo4j}

\section{Język Cypher}

\chapter{Sposoby dostępu do neo4j}

\section{Embedded mode}

\section{REST mode}

\section{Zapytania Cypher}

\chapter{Dostęp do bazy poprzez DSL}

\section{Wstęp}

\section{Bazy SQL}

\subsection{Squeryl}

\section{Bazy NoSQL}

\subsection{Mongoid}

\chapter{DSL do baz grafowych}

\section{Budowanie wzorców}

\subsection{Wstęp}

Zapytania w bazach grafowych skupiają się na wyszukiwaniu określonych wzorców w grafie. Z tego powodu główny nacisk w obrębie ekspresywności tworzonego DSLa został położony właśnie na tej części.

Stworzone narzędzie umożliwia defininiowanie wzorców na dwa niezależne sposoby.

Pierwszy z nich polega na składaniu wywołań funkcji
\lstinputlisting{listings/scala/dsl/patterns/functional-example-1.scala}

Drugi sposób do złudzenia przypomina tworzenie listy 
\lstinputlisting{listings/scala/dsl/patterns/lists-like-example-1.scala}

\section{Nakładanie warunków na wyniki zapytań}

\section{Serializacja do języka Cypher}

\section{Zwracanie wyników}

% zakończenie 
\summary
Podsumowanie

% załączniki (opcjonalnie):
\appendix
\chapter{Tytuł załącznika jeden}

Treść załącznika jeden.

% literatura (obowiązkowo):
\bibliographystyle{unsrt}
\bibliography{master-thesis}

% spis tabel (jeżeli jest potrzebny):
\listoftables

% spis rysunków (jeżeli jest potrzebny):
\listoffigures

\oswiadczenie

\end{document}
